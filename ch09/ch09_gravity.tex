\documentclass[12pt]{article}
\usepackage{graphicx}

\graphicspath{{images/}}

\title{NCERT Physics Questions \\ Chapter 9: Gravitation \\ Solutions}
\author{}
\date{}

\begin{document}

\maketitle

\section*{Theory questions}
\subsection*{Short theory questions}

\begin{enumerate}

	\item What is centripetal force? \\
		{\itshape Ans:
			\begin{itemize}
				\item[-] Ans: The force that causes the acceleration to keep the body moving
			along the circular path is centripetal forces.
				\item[-] It acts towards the center of the circle, along the radius.
			\end{itemize}
		}

	\item Give a basic definition of gravitational force. \\
		{\itshape Ans:
			All objects in the universe attract each other.
			This force of attraction between objects is
			called the gravitational force.
		}
	\item Describe the universal law of gravitation. \\
		{\itshape Ans: According to the universal law of gravitation,
			the gravitational force between two objects is
			\begin{itemize}
				\item[-] directly proportional to the product of their masses,
				\item[-] inversely proportional to the distance between them,
				\item[-] and acts along the line joining the \textbf{center} of the two objects.
			\end{itemize}
		}
	\item What is the value of gravitational constant G and give it's SI unit. \\
		{\itshape
			\begin{itemize}
				\item[-] Value is $6.673 \times 10^{-11} Nm^2{kg}^{-2}$
				\item[-] It's SI unit is $Nm^2{kg}^{-2}$
			\end{itemize}
		}
	\item Define freefall. \\
		{\itshape Ans:
			Whenever objects fall
			towards the earth under this force alone, we
			say that the objects are in free fall.
		}
	\item Give two notable properties of gravitational acceleration on Earth. \\
		{\itshape Ans:
			\begin{itemize}
				\item[-] The magnitiude of acceleration is independent of mass of the object.
				\item[-] It's value is constant near the earth.
			\end{itemize}
		}
	\item Define weight and its SI unit. \\
		{\itshape Ans:
			\begin{itemize}
				\item[-] Weight of an object is the force with which 
					it is attracted towards the earth.
				\item[-] It's SI unit is Newton (N), same as that of force.
			\end{itemize}
		}
	\item What is the difference between mass and weight? \\
		{\itshape Ans:
			\begin{itemize}
				\item[-] Mass is the quantity of matter in an object, while weight is the force with which it is attracted towards the earth.
				\item[-] Mass of an object remains same everywhere on Earth, while weight changes slightly with location depending on value of g.
				\item[-] Mass has no direction, while weight is a force and has a direction.
			\end{itemize}
		}
	\item What is the ratio of gravitational acceleration on the moon and on Earth? \\
		{\itshape Ans: $\frac{g_{moon}}{g_{earth}} = \frac{1}{6} $}
	\item Define thrust. \\
		{\itshape Ans:
			The force acting on an object perpendicular to
			the surface is called thrust.
		}
	\item Define pressure. \\
		{\itshape
			\begin{itemize}
				\item[-] The thrust on unit area is called pressure.
				\item[-] It's SI unit is Pascal (Pa) or Nm$^{-2}$.
			\end{itemize}
		}
	\item State two characteristics of pressure in fluids. \\
		{\itshape
			\begin{itemize}
				\item[-] Fluids exert pressure on the base and walls of the
				container in which they are enclosed. 
				\item[-] Pressure exerted in any confined mass of fluid is
				transmitted undiminished in all directions.
			\end{itemize}
		}
	\item Define buoyant force. \\
		{\itshape Ans:
			The \textbf{upward} force exerted by the water on
			the bottle is known as upthrust or buoyant
			force.
		}
	\item Define density. \\
		{\itshape Ans:
			The mass per unit volume of a substance is called density.
		}
	\item State the relation between density and sinking of an object in a fluid. \\
		{\itshape Ans:
			Objects of density less than that of a liquid float
			on the liquid. The objects of density greater
			than that of a liquid sink in the liquid.
		}
	\item State Archimedes principle. {\itshape(You should state it exactly in roughly the same words)} \\
		{\itshape Ans:
			When a body is immersed fully or partially
			in a fluid, it experiences an \textbf{upward} force that
			is equal to the \textbf{weight of the fluid displaced}
			by it.
		}

\end{enumerate}

\subsection*{Long theory questions}
\begin{enumerate}
	\item Derive the formula of gravitational force between two objects.
	\item Derive the SI unit of gravitational constant.
	\item Derive the formula and calculate the value of acceleration of gravity on Earth.
	\item Derive and find the ratio of gravitational constant on the moon and on Earth.
\end{enumerate}

\end{document}
