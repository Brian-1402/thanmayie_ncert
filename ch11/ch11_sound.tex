\documentclass[12pt]{article}
\usepackage{graphicx}
% \usepackage[section]{placeins}
% \usepackage[a4paper,margin=0.75in]{geometry}
% \usepackage{array}
% \usepackage{varwidth}
% \usepackage{tabularx}
% \usepackage{amsmath}
% \usepackage{subcaption}
% \usepackage{svg}

\graphicspath{{images/}}

\title{NCERT Physics Questions \\ Chapter 11: Sound}
\author{}
\date{}

\begin{document}

\maketitle

\section*{Theory questions}
\subsection*{Short theory questions}
\begin{enumerate}
	\item What is a medium for sound? \\
		\textit{Ans: The matter or substance through which sound
			is transmitted is called a medium.}
	\item What is a wave in a medium? \\
		\textit{Ans: A wave is a disturbance that moves
			through a medium when the particles of the
			medium set neighbouring particles into motion.}
	\item What is the meaning of a mechanical wave? Give example of \\ mechanical and non mechanical wave. \\
		\textit{Ans: Waves characterised by the
			motion of particles in the medium are
			called mechanical waves. \\
			Example: mechanical-sound, non-mechanical-light.}
	\item Give the three characteristics of a sound wave. \\
		\textit{Ans: frequency, amplitude, and speed.}
	\item Define wavelength and give its SI unit. \\
		\textit{Ans: The distance between two consecutive
			compressions (C) or two consecutive
			rarefactions (R) is called the wavelength. \\
			Its SI unit is metre (m).}
	\item Formal definition of one oscillation of a sound wave. \\
		\textit{Ans: The change in density/pressure/periodic wave property
			from the maximum value to the minimum value, then again to the
			maximum value, makes one complete oscillation.}
	\item Define frequency and give its SI unit. \\
		\textit{Ans: The number of such oscillations of the wave
			per unit time is the frequency of the sound wave. }
	\item What is time period of a wave. \\
		\textit{Ans: The time taken for one complete oscillation 
			is called the time period of the sound wave.}
	\item Formula relation between frequency and time period of a wave. \\
		\textit{Ans: $ v = \frac{1}{T} $}
	\item What is pitch of a sound wave? \\
		\textit{Ans: How the brain interprets the frequency
			of an emitted sound is called its pitch.}
	\item Define amplitude of a wave and its unit. \\
		\textit{Ans: The magnitude of the maximum
			disturbance in the medium on either side of
			the mean value is called the amplitude of the
			wave. For sound its unit will be that of 
			either density ($kg/{m}^{3}$) or pressure (Pascal).}
	\item What is loudness? What does the definition of loudness and pitch have in common? \\
		\textit{Ans: Loudness is a measure of the response
			of the ear to the sound. In other words, it is how the
			brain interprets amplitude of sound, for a particular frequency. \\
			Both loudness and pitch are subjective observations of 
			the brain and ears, and not a properly measurable quantity.}
	\item What is a tone and a note? How is it different from noise? \\
		\textit{Ans: 
			\begin{itemize}
				\item A sound of single frequency is called a tone. 
				\item The sound which is produced due to a mixture of several frequencies is called a note and is pleasant to listen to.
				\item Noise is a sound of mixed frequencies which is \textbf{unpleasant} to listen to.
			\end{itemize} }
	\item Another name for sound quality. Also give an informal definition of quality of a sound. \\
		\textit{Ans: timber of sound. The sound which is more
		pleasant is said to be of a rich quality.}
	\item Define speed of sound. \\
		\textit{Ans: The distance which a point on a wave, such as a
			compression or a rarefaction, travels per unit time.}
	\item Give the formula of speed of sound using a) Time period, b) frequency. \\
		\textit{Ans:
		a) speed = $ \frac{\lambda}{T} $
		b) speed = $ \lambda \cdot \nu $}
	\item Define intensity of sound. \\
		\textit{Ans: The amount of sound energy passing each
			second \textbf{through unit area} is called the intensity
			of sound.}
	\item When does the speed of sound remain same?  \\
		\textit{Ans: The speed of sound remains almost the
			same for all frequencies in a given medium
			under the same physical conditions.}
	\item Give the two properties that speed of sound depends on. \\
		\textit{Ans:
			\begin{itemize}
				\item Temperature of the medium,
				\item The state of the medium, or the medium itself.
			\end{itemize}
			}
	\item State the law of reflection of sound. \\
		\textit{Ans: The directions in which the sound is incident and is reflected
			make equal angles with the normal to the
			reflecting surface at the point of incidence, and
			the three are in the same plane.}
	\item What is the time delay between sound reflections for our brains to register it as a separate echo? \\
		\textit{Ans: 0.1s}
	\item What is reverberation and how do we reduce it? \\
		\textit{Ans: The repeated reflection that
			results in this persistence of sound is
			called reverberation. To reduce reverberation, the
			roof and walls of the auditorium are
			generally covered with sound-absorbent
			materials.}
	\item Audible range of sound for the average human. \\
		\textit{Ans: 20Hz to 20kHz.}
	\item What is infrasound and ultrasound, and give some real world examples for each. \\
		\textit{Ans:  Sounds of frequencies below 20 Hz are called infrasonic sound or
			infrasound. Example: Rhinoceros communicating. \\
			Frequencies higher than 20 kHz are called ultrasonic sound or ultrasound.
			Example: Echolocation in bats.}
	\item What is echocardiography? \\
		\textit{Ans: Ultrasonic waves are made to reflect
			from various parts of the heart and
			form the image of the heart. This technique is called echocardiography.}
	\item Explain ultrasonography. \\
		\textit{Ans: In this technique the ultrasonic waves
			travel through the tissues of the body
			and get reflected from a region where
			there is a change of tissue density.
			These waves are then converted into
			electrical signals that are used to
			generate images of the organ. These
			images are then displayed on a monitor
			or printed on a film. This technique
			is called ‘ultrasonography’.}
\end{enumerate}

\subsection*{Long theory questions}
\begin{enumerate}
	\item Explain the propogation of sound in terms of
	\begin{enumerate}
		\item its effect on each particle in the medium.
		\item pressure differences in the medium.
	\end{enumerate}
	\item Definition and difference between longtitudinal wave and transverse wave.
	\item Define the peak and trough of a wave.
	\item Give the difference between amplitude, intensity, and loudness of a sound wave. Are they related to one another?
	\item Give 3 uses of reflection of sound to our advantage.
	\item \textit{(very big)} Briefly explain each of the five applications of ultrasound.
\end{enumerate}

% \section*{Statement problems}
% \begin{enumerate}
% \end{enumerate}

% \section*{Numericals}
% \subsection*{Hard}
% \begin{enumerate}
% \end{enumerate}

\end{document}
