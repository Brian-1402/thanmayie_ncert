\documentclass[12pt]{article}
\usepackage{graphicx}
% \usepackage[section]{placeins}
% \usepackage[a4paper,margin=0.75in]{geometry}
% \usepackage{array}
% \usepackage{varwidth}
% \usepackage{tabularx}
% \usepackage{amsmath}
% \usepackage{subcaption}
% \usepackage{svg}
% \usepackage{tikz}

\graphicspath{{images/}}

\title{NCERT Physics Questions \\ Chapter 7: Motion \\ Solutions}
\author{}
\date{}

\begin{document}

\maketitle

\section*{Theory questions}
\subsection*{Short theory questions}
\begin{enumerate}
	\item Define displacement. \\
		\textit{Ans: The shortest distance measured from the initial to the final
			position of an object is known as the displacement.}
	\item Define uniform motion.  \\
		\textit{Ans: When object covers equal distances in equal
		intervals of time, it is said to be in uniform motion.}
	\item Define speed. Also define average speed. \\
		{\itshape Ans: (Best to also mention SI unit and formula in definitions.)
			\begin{itemize}
				\item[-] Speed is distance travelled by the object in unit time. 
				\item[-] SI unit: metre per second, m/s or $m \; s^{-1}$. 
				\item[-] (Explain the symbols in the formula) If an object travels a distance s in time t then its speed $ v = \frac{s}{t} $.
				\item[-] Average speed is the total distance travelled by the object divided by the total time taken.
			\end{itemize}
		}
	\item Does speed have direction? \\
		\textit{Ans: No. }
	\item Define velocity. Does it have direction?  \\
		{\itshape Ans:
			\begin{itemize}
				\item[-] Velocity is the speed of an object moving in a definite direction.
				\item[-] Yes, velocity has direction.
			\end{itemize}
		}
	\item Give two definitions of average velocity. \\
		{\itshape Ans:
			\begin{itemize}
				\item[-] Definition 1: Arithmetic mean of initial velocity and final velocity for a given period of time.
				\item[-] $v_{av} = \frac{u + v}{2}$
				\item[-] Definition 2: Total displacement divided by the total time taken.
				\item[-] $v_{av} = \frac{total \; displacement}{total \; time}$
				\item[-] It has same unit of speed, m/s.
			\end{itemize}
		}
	\item State differences between speed and velocity. \\
		{\itshape Ans:
			\begin{itemize}
				\item[-] Speed has no direction, velocity has direction.
				\item[-] Avg speed is distance by time, avg velocity is displacement by time.
			\end{itemize}
		}
	\item Define acceleration. \\
		{\itshape Ans:
			\begin{itemize}
				\item[-] Change in the velocity of an object per unit time.
				\item[-] If the velocity of an object changes from an initial value u
					to the final value v in time t, the acceleration a is
					$ a = \frac{v - u}{t} $.
				\item[-] SI unit: $ m \; s^{-2} $.
			\end{itemize}
		}
	\item Define uniform acceleration. \\
		{\itshape Ans: An object is in uniform acceleration if 
			\begin{itemize}
				\item[-] The object is moving in a straight line,
				\item[-] its velocity increases or decreases by equal
					amounts in equal intervals of time.
			\end{itemize}
		}
	\item Give examples of uniform and non uniform acceleration. \\
		{\itshape Ans:
			Uniform acceleration:
			\begin{itemize}
				\item[-] Object falling freely under gravity.
			\end{itemize}
			Non-uniform acceleration:
			\begin{itemize}
				\item[-] A car travelling along a straight road increases its speed by unequal amounts in equal intervals of time.
			\end{itemize}
		}
	\item Describe the values of speed, velocity and acceleration for uniform motion.
		{\itshape Ans: 
			\begin{itemize}
				\item Speed is constant.
				\item Velocity is constant.
				\item Magnitude of velocity is equal to speed.
				\item Acceleration is zero.
			\end{itemize}
		}
	\item Describe the shape of the distance-time graph for an object in a) uniform motion, b) uniform acceleration, c) zero speed. \\
		{\itshape Ans: 
			\begin{enumerate}
				\item Straight line with no-zero slope.
				\item Curved line.
				\item Straight horizontal line parallel to time axis.
			\end{enumerate}
		}
	\item Describe the shape of the velocity-time graph foran object in a) uniform motion, b) uniform acceleration, c) zero speed. \\
		{\itshape Ans: 
			\begin{enumerate}
				\item Straight horizontal line parallel to time axis.
				\item Straight line with non-zero slope.
				\item Straight horizontal line exactly on the time axis.
			\end{enumerate}
		}
	\item How to get value of a) acceleration, b) displacement from velocity-time graph? \\
		{\itshape Ans: 
			\begin{enumerate}
				\item Slope of the graph.
				\item Area under the graph.
			\end{enumerate}
		}
	\item How to get value of speed from distance-time graph? \\
		{\itshape Ans: Slope of the graph.}
	\item Give the 3 equations of motion. In which situations of motion do they work? \\ 
		{\itshape Ans: 
			\begin{enumerate}
				\item $ v = u + at $
				\item $ s = ut + \frac{1}{2}at^2 $
				\item $ 2as = v^2 - u^2 $
			\end{enumerate}
		where $ s $ is displacement, $ u $ is initial velocity, $ v $ is final velocity, $ a $ is uniform acceleration, and $ t $ is time. (Don't forget to define the meanings of the symbols while answering.) \\
		These formulas work only for uniform acceleration along a straight line.

		}
	\item Define uniform circular motion. \\
		{\itshape Ans: An object is said to be in uniform circular motion if 
			\begin{itemize}
				\item[-] it moves in a circular path with constant speed,
				\item[-] the only change in velocity is the direction and the magnitude is constant.
			\end{itemize}
			Formula of speed: $ v = \frac{2 \pi r}{T} $, where $ r $ is radius of the circle, and $ T $ is time period.
	}
	\item Give 3 properties of uniform circular motion. \\
		{\itshape Ans:
			\begin{itemize}
				\item[-] Speed is constant,
				\item[-] Direction of velocity at any moment is tangential to the circle.
				\item[-] Rate of change of velocity is constant, in direction only (magnitude is constant),
			\end{itemize}
		}
\end{enumerate}


\end{document}
