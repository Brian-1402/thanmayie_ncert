\documentclass[12pt]{article}
\usepackage{graphicx}

\graphicspath{{images/}}

\title{NCERT Physics Questions \\ Chapter 8: Forces and Laws of Motion \\ Solutions}
\author{}
\date{}

\begin{document}

\maketitle

\section*{Theory questions}
\subsection*{Short theory questions}

\begin{enumerate}

	\item State the first law of motion. \\
		{\itshape
			Ans: An object remains in a state of rest or of
			uniform motion in a straight line unless
			compelled to change that state by an
			applied force.
		}

	\item Define inertia. \\
		{\itshape 
			Ans: The tendency of undisturbed objects to stay
			at rest or to keep moving with the same
			velocity is called inertia.
		}

	\item What is the relation between inertia and mass. \\
		{\itshape Ans:
			\begin{itemize}
				\item[-] The mass of an object is a measure of
				its inertia.
				\item[-] Inertia is a qualitative property, mass is the quantitative measure of inertia.
			\end{itemize}
		}

	\item Define 2nd law of motion. \\
		{\itshape Ans:
			The second law of motion states that the
			rate of change of momentum of an object is
			proportional to the applied unbalanced force
			in the direction of force. \\ 
			Its numerical formula is \[ F \propto \frac{p_2 - p_1}{t} \]
			\[ F \propto \frac{m(v-u)}{t} \]
			where an object of mass, m is moving along
			a straight line with an initial velocity, u. It is
			uniformly accelerated to velocity, v in time, t
			by the application of a constant force, F
			throughout the time, t. The initial and final
			momentum of the object will be, p1 = mu and p2 = mv respectively.
		}

	\item Define the SI unit of force. \\
		{\itshape
			Ans:  One unit of force is $ kg \; m \; s^{-2} $ or newton N.
			It is defined as the amount that produces an 
			acceleration of 1 m s-2 in an object of 1 kg mass.
		}

	\item Define Newton's 3rd law of motion. \\
		{\itshape
			\begin{itemize}
				\item
			\begin{itemize}
				\item The third law of motion states that when one
				object exerts a force on another object, the
				second object instantaneously exerts a force
				back on the first. \\ Alternatively,
				\item To every action there is an equal
					and opposite reaction
			\end{itemize}
				\item The action and reaction always act on two different objects, simultaneously.
			\end{itemize}
		}

\end{enumerate}

\subsection*{Long theory questions}
\begin{enumerate}
	\item Derive equation of force from Newton's 2nd law of motion. \\
		{\itshape Ans:
			Consider an object of mass, m is moving along
			a straight line with an initial velocity, u. It is
			uniformly accelerated to velocity, v in time, t
			by the application of a constant force, F
			throughout the time, t. The initial and final
			momentum of the object will be, p1 = mu and p2 = mv respectively. \\
			The change in momentum:
			\[\propto p_2 - p_1\]
			\[\propto mv - mu\]
			\[\propto m \times (v - u)\]
			The rate of change of momentum: \[ \propto  \frac {m \times (v - u)}{t} \]
			Using Newton's 2nd law, 
			\[ F \propto \frac{m(v-u)}{t} \]
			\[ F = \frac{km(v-u)}{t} \]
			Using $ a = \frac{v-u}{t} $, 
			\[ F = kma \]
			Units of F are chosen such that k = 1. Therefore,
			\[ F = ma \]
		}
\end{enumerate}

\end{document}
