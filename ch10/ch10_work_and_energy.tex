\documentclass[12pt]{article}
\usepackage{graphicx}
% \usepackage[section]{placeins}
% \usepackage[a4paper,margin=0.75in]{geometry}
% \usepackage{array}
% \usepackage{varwidth}
% \usepackage{tabularx}
% \usepackage{amsmath}
% \usepackage{subcaption}
% \usepackage{svg}
% \usepackage{tikz}

\graphicspath{{images/}}

\title{NCERT Physics Questions \\ Chapter 10: Work and Energy}
\author{}
\date{}

\begin{document}

\maketitle

\section*{Theory questions}
\subsection*{Short theory questions}
\begin{enumerate}
	\item Define work using both the formula and the sentence definition.
	\item Mention the SI unit of work (both standalone unit, and in terms of work's constituents from the formula).
	\item Give the definition of SI unit of work.
	\item Does work have direction? Can it be negative? Does it depend on the exact path taken?
	\item Mention the relation between work and energy (in one point).
	\item What are the two fundamental types of energy?
	\item Give the definition of kinetic energy.
	\item Define potential energy. Give a formula relation between potential energy and work done by an agent.
	\item Define gravitational potential energy.
	\item Define power and its formula. Does it have direction?
	\item Mention the SI unit of power (both standalone unit, and in terms of power's constituents from the formula).
	\item \textit{(extra)} Difference between average power and instantaneous power.
	\item For problem solving purposes, can we apply conservation of energy formula when friction is involved? If no, how do we deal with energy lost by friction?
	% \item \textit{(extra, introduces vectors)} Can power be negative? If work can be negative, does that imply it can have direction? And similarly for power?
\end{enumerate}

\subsection*{Long theory questions}
\begin{enumerate}
	\item Derive the formula of kinetic energy. \\ \textit{Steps: a) State the variables and the setup, b) Mention the formulas to be used, c) Interconnect the formulas to get final result.}
	\item Derive the formula for gravitational potential energy.
	\item State the law of conservation of energy in 3 points. Give the formula in terms of kinetic and potential energy.
\end{enumerate}

\section*{Statement problems}
\begin{enumerate}
	\item A person is holding a ball on a string and it is swinging like a pendulum steadily. 
	\begin{enumerate}
		\item Does the person holding the string do any work at any given moment? 
		\item \textit{(extra)} What are the moments in the motion of pendulum where gravity is doing zero instantaneous power?
	\end{enumerate}
\end{enumerate}

\section*{Numericals}
% \subsection*{Hard}
\begin{enumerate}
	\item A 1kg block is launched sliding on a rough surface at 20m/s and reaches a stop in 2 seconds. What is the total work done by friction? 
	\item What is the work done by a clock between 12pm and 6pm? The length of the hour hand is 10cm and all its mass is focused at the tip, weighing 10g. Also calculate the average power of the clock.
\end{enumerate}

\end{document}
